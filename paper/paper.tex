%%%%%%%%%%%%%%%%%%%%%%%%%%%%%%%%%%%%%%%%%%%%%%%%%%%%%%%%%%%%%%%%%%%%%%
% How to use writeLaTeX:
%
% You edit the source code here on the left, and the preview on the
% right shows you the result within a few seconds.
%
% Bookmark this page and share the URL with your co-authors. They can
% edit at the same time!
%
% You can upload figures, bibliographies, custom classes and
% styles using the files menu.
%
%%%%%%%%%%%%%%%%%%%%%%%%%%%%%%%%%%%%%%%%%%%%%%%%%%%%%%%%%%%%%%%%%%%%%%

\documentclass[12pt]{article}
\usepackage{float}
\usepackage{booktabs}
\usepackage{tabulary}
\usepackage{svg}
\usepackage{tikz}
\usepackage{pgfplots}
\usepackage{indentfirst} %alinhar 1 parágrafo

\usepackage{sbc-template}
\usepackage{graphicx}
\usepackage{dirtytalk}

\usepackage[T1]{fontenc}
\usepackage{mathtools}
\usepackage{blkarray, bigstrut}
\usepackage{gauss}
\usepackage{amsmath}
\usepackage{afterpage}
\usepackage{graphicx,url}
\usepackage{subcaption}

%\usepackage[brazil]{babel}
\usepackage[utf8]{inputenc}

\pgfplotsset{compat=1.10}
\usetikzlibrary{%
    intersections,%
    arrows,%
    decorations.pathmorphing,%
    backgrounds,%
    positioning,%
    fit,%
    petri,%
    calc,%
    through,%
    graphs,%
    shapes.misc,%
    trees,%
    mindmap,%
    shadows,%
    calendar%
}

\sloppy

\title{Assessment of climate change impacts on hydro resources in Três Marias reservoir using machine learning models}

\author{%
    Bruno T. R. V. Silva\inst{1}, %
    Carolina S. Ansélmo\inst{1}, \\%
    Paula C. S. Borba\inst{1}, %
    Wallace S. S. Souza\inst{1}%
}

\address{%
    Department of Civil Engineering -- Aeronautics Institute of Technology\\
    São José dos Campos, Brazil
    \email{bruno.valeriano@ga.ita.br, carolina.anselmo@ga.ita.br}%
    \vspace{-2ex}
    \email{paula.borba@ga.ita.br,  wallace.souza@ga.ita.br}
}

\begin{document}

\maketitle

%\begin{resumo}

%\end{resumo}

\section{Introduction}



\begin{figure}[htbp]
    {\centering
    \begin{subfigure}[b]{.49\textwidth}
        \centering
        \includesvg{graphs/R2_w01}
        \caption{<caption>}
    \end{subfigure}
    \begin{subfigure}[b]{.49\textwidth}
        \centering
        \includesvg{graphs/R2_w15}
        \caption{<caption>}
    \end{subfigure}}

    \begin{subfigure}[b]{.49\textwidth}
        \centering
        \includesvg{graphs/R2_w30}
        \caption{<caption>}
    \end{subfigure}
    \caption{<caption>}
    \label{<label>}
\end{figure}

\begin{figure}[htbp]
    {\centering
    \begin{subfigure}[b]{.49\textwidth}
        \centering
        \includesvg{graphs/MAE_w01}
        \caption{<caption>}
    \end{subfigure}
    \begin{subfigure}[b]{.49\textwidth}
        \centering
        \includesvg{graphs/MAE_w15}
        \caption{<caption>}
    \end{subfigure}}

    \begin{subfigure}[b]{.49\textwidth}
        \centering
        \includesvg{graphs/MAE_w30}
        \caption{<caption>}
    \end{subfigure}
    \caption{<caption>}
    \label{<label1>}
\end{figure}


%\subsection{Machine learning models} \label{subsec:pre-processamento}

%Parte a ser entregue até o dia 21/06.

%\section{Discussion}

%Discutir os resultados obtidos. Parte a ser entregue até o dia 21/06.

%\section{Conclusion}

\bibliographystyle{sbc}
\bibliography{sbc-template}

\end{document}
